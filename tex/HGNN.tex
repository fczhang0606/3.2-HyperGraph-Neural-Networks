%%%%%%%%%%%%%%%%%%%%%%%%%%%%%%%%%%%%%%%%%%%%%%%%文档类型
\documentclass{article}


%%%%%%%%%%%%%%%%%%%%%%%%%%%%%%%%%%%%%%%%%%%%%%%%引入宏包
\usepackage[fleqn]{amsmath}  % https://zhuanlan.zhihu.com/p/464170020
\usepackage{amssymb}
\usepackage{amsthm}
\usepackage{mathtools}

\usepackage{geometry}
%\geometry{a4paper, landscape}  % 设置A4纸张并转为横向模式
\usepackage{CJKutf8}

\usepackage{booktabs}  % 三线表


%%%%%%%%%%%%%%%%%%%%%%%%%%%%%%%%%%%%%%%%%%%%%%%%正文内容
\begin{document}


%%%%%%%%%%%%%%%%%%%%%%%%%%%%%%%%%%%%%%%%%%%%%%%%
\section*{ch1: Hypergraph}


~ \\[3pt]
\begin{CJK}{UTF8}{gbsn}
    理解: \\[3pt]
\end{CJK}


%
\begin{align*}
    & \mathcal{G} = ( \mathcal{V} , \mathcal{E} , W ) \\[3pt]
        & \qquad \mathcal{G} \quad \text{: hypergraph} \\[3pt]
        & \qquad \mathcal{V} \quad \text{: vertices} 
          \qquad \qquad \text{U: Vertex Weight Matrix} \\[3pt]
        & \qquad \qquad \qquad \qquad \qquad \quad \ \ \text{X: Vertex Feature Matrix} 
          \qquad \text{Y: Vertex Label Matrix} \\[3pt]
        & \qquad \mathcal{E} \quad \text{: hyperedges} 
          \qquad \; \text{W: Hyperedge Weight Matrix} \\[3pt]
\end{align*}

%
\begin{align*}
    & H \in { |\mathcal{V}| * |\mathcal{E}| } \qquad \qquad 
    H(v, e) = \left\{ 
        \begin{array}{cc}
            1       & \text{if} \ v \in e    \\[3pt]
            0       & \text{if} \ v \notin e \\[3pt]
        \end{array} \right. 
\end{align*}

%
\begin{align*}
    & d(v) = \sum_{e \in \mathcal{E}} H(v, e)*w(e) \qquad D_{v} \\[3pt]
    & d(e) = \sum_{v \in \mathcal{V}} H(v, e)      \qquad \qquad \quad D_{e} \\[3pt]
\end{align*}

%
\begin{align*}
    & \Delta = D_{v} - H W D_{e}^{-1} H^{T} \\[3pt]
    & \Delta = I - D_{v}^{-1/2} H W D_{e}^{-1} H^{T} D_{v}^{-1/2} \\[3pt]
\end{align*}


%%%%%%%%%%%%%%%%%%%%%%%%%%%%%%%%%%%%%%%%%%%%%%%%
\newpage
\section*{ch2: Spectral Hypergraph Theory}


\begin{align*}
    ~ \\[3pt]
    g \star x &= \phi ( (\phi^{T} g) \odot (\phi^{T} x) ) 
              = \phi g(\wedge) (\phi^{T} x)  \\[3pt]
    g(\wedge) &= diag( g(\lambda_{1}), ..., g(\lambda_{n}) )  \\[3pt]
    ~ \\[3pt]
    g \star x & \approx \sum_{k=0}^{K} \theta_{k} T_{k} ( \tilde{\Delta} ) x  \\[3pt]
    \tilde{\Delta} & = \frac{2}{\lambda_{max}} \Delta - I  \\[3pt]
    ~ \\[3pt]
    K & = 2  \qquad \qquad \lambda_{max} = 2  \\[3pt]
    g \star x & \approx \theta_{0}x - \theta_{1} 
        D_{v}^{-1/2} H W D_{e}^{-1} H^{T} D_{v}^{-1/2} x  \\[3pt]
    ~ \\[3pt]
    \theta_{0} &= (1/2) \theta D_{v}^{-1/2} H D_{e}^{-1} H^{T} D_{v}^{-1/2} \qquad \qquad 
    \theta_{1} = (-1/2) \theta  \\[3pt]
    g \star x & \approx (1/2) \theta D_{v}^{-1/2} H (W+I) D_{e}^{-1} H^{T} D_{v}^{-1/2} x  \\[3pt]
              & \approx \theta D_{v}^{-1/2} H W D_{e}^{-1} H^{T} D_{v}^{-1/2} x  \\[3pt]
    ~ \\[3pt]
    X^{t+1} &= \sigma ( D_{v}^{-1/2} H W D_{e}^{-1} H^{T} D_{v}^{-1/2} X^{t} \varTheta )  \\[3pt]
\end{align*}

%%%%%%%%%%%%%%%%%%%%%%%%%%%%%%%%%%%%%%%%%%%%%%%%
\newpage
\section*{ch3: Hypergraph Generation and Transformation}


~ \\[3pt]
\begin{CJK}{UTF8}{gbsn}
    理解: \\[3pt]
        隐式:距离、特征  \\[3pt]
        显式:属性、网络  \\[3pt]
\end{CJK}


%%%%%%%%%%%%%%%%%%%%%%%%%%%%%%%%%%%%%%%%%%%%%%%%
\newpage
\section*{ch3: Hypergraph Learning Architecture}


~ \\[3pt]
\begin{CJK}{UTF8}{gbsn}
    理解: \\[3pt]
        超图游走  \\[3pt]
\end{CJK}


~ \\[3pt]
(1) Features
\begin{align*}
    & X \in R^{ |V| \times d } \qquad \qquad Y \in R^{ |E| \times d'} \\[3pt]
    & \text{Externel + Internal(local+global) + Identity}
\end{align*}


~ \\[3pt]
(2) Transformation

~ \\[3pt]
Reductive Transformation
\begin{align*}
    & ( E, X, Y ) \Rightarrow A 
    \qquad \qquad 
    \text{hyperedges to edges} \\[3pt]
    & \text{clique expansion + adaptive expansion}
\end{align*}

~ \\[3pt]
Non-reductive Transformation
\begin{align*}
    & \text{star/line/tensor expansion}
\end{align*}


~ \\[3pt]
(3) Message
\begin{align*}
    & \text{whose: v-v \ v-e \ e-v} \\[3pt]
    & \text{what : e-consistent + e-dependent} \\[3pt]
    & \text{how  : fixed-pooling + learnable-pooling} 
\end{align*}


~ \\[3pt]
(4) Training


\end{document}

